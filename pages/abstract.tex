\chapter{\abstractname}

\textbf{\textit{TODO: KEEP UPDATING ABSTRACT WHILE WRITING}}


This thesis introduces a novel method for the automated annotation of mathematical identifiers in scientific papers, leveraging \ac{LLMs} such as GPT-3.5 and GPT-4. The approach addresses the challenges of co-reference resolution and formula grounding, traditionally handled by human annotators through costly and time-intensive procedures. Our study utilises a \ac{MioGatto}, and explores the potential of integrating \ac{POS} tagging and other technologies assisting in the process.
The key component of this research is the development of a procedure for generating a dictionary of mathematical identifiers, contextualising their various meanings, and enabling the language model to select the most accurate definition based on the given context. This method demonstrates the impressive capability of LLMs to disambiguate meanings based on context, a vital task due to the inherently polysemous nature of mathematical identifiers.
The preliminary results of \ac{CoNLL} scores of 82.36 position our approach as a potential game-changer in mathematical text annotation, offering a significant reduction in time and financial costs.
The findings underscore the promise of the untapped potential of general purpose Large Language Models in specific mathematical language understanding.

%TODO: Acknowledge
%TODO: Acknowledge supervisors, advisors, family, friends.