\chapter{Appendix}

This chapter contains all the information that we deemed to be not crucial for the main content.

\lstset{
  literate={∫}{\textbackslash int}{1}
           {ε}{\textbackslash epsilon}{1}
           {→}{\textbackslash to}{1}
           {μ}{\textbackslash mu}{1},
  basicstyle=\ttfamily
}

\section{Definitions}
\label{app:definitions}

\begin{definition}[Token]
\label{def:token}
A token is an instance of a sequence of characters in some particular document that are grouped together as a useful semantic unit for processing. \footnote{\url{https://nlp.stanford.edu/IR-book/html/htmledition/tokenization-1.html}}
\end{definition}

\begin{definition}[Prompt]
\label{def:prompt}
A prompt is a question, statement, or request that is input into an AI system to elicit a specific response or output.\footnotemark
\end{definition}
\footnotetext{\href{https://dev.to/avinashvagh/understanding-the-concept-of-natural-language-processing-nlp-and-prompt-engineering-35hg}{https://dev.to/avinashvagh/understanding-the-concept-of-natural-language-processing-} \\ \href{https://dev.to/avinashvagh/understanding-the-concept-of-natural-language-processing-nlp-and-prompt-engineering-35hg}{nlp-and-prompt-engineering-35hg}}



\section{XML Encoding Example}
This section shows the complicated formatting of XML which renders it as an unsuitable type for input to \ac{LLMs}.

\subsection{Quadratic Equation}
\label{app:xml_quadratic}

The XML encoding of $x = \frac{-b \pm \sqrt{b^2 - 4ac}}{2a}$ is as follows:
\begin{verbatim}
<math display="block" style="display:block math;">
  <mrow>
    <mi>x</mi>
    <mo>=</mo>
    <mfrac>
      <mrow>
        <mo>-</mo>
        <mi>b</mi>
        <mo>±</mo>
        <msqrt>
          <mrow>
            <msup>
              <mi>b</mi>
              <mn>2</mn>
            </msup>
            <mo>-</mo>
            <mn>4</mn>
            <mi>a</mi>
            <mi>c</mi>
          </mrow>
        </msqrt>
      </mrow>
      <mrow>
        <mn>2</mn>
        <mi>a</mi>
      </mrow>
    </mfrac>
  </mrow>
</math>
\end{verbatim}

\subsection{Ampere's Circuit Law}
\label{app:xml_ampere}

The XML encoding of 
$ \oint_C \vec{B}.d\vec{\ell} = \mu_0 (I_{enc} 
+ \epsilon_0 \frac{d}{d_t}  {\int_S \vec{E}.\hat{n} \, da ) }$
is as follows:
\begin{verbatim}
<math>
  <mrow>
    <msub>
      <mo movablelimits="false">∮</mo>
      <mi>C</mi>
    </msub>
    <mover>
      <mi>B</mi>
      <mo stretchy="false" style="transform:scale(0.75) 
      translate(10%, 30%);">→</mo>
    </mover>
    <mo>∘</mo>
  </mrow>
  <mrow>
    <mrow>
      <mi mathvariant="normal">d</mi>
    </mrow>
    <mover>
      <mi>l</mi>
      <mo stretchy="false" style="transform:scale(0.75) 
      translate(10%, 30%);">→</mo>
    </mover>
    <mo>=</mo>
  </mrow>
  <mrow>
    <msub>
      <mi>μ</mi>
      <mn>0</mn>
    </msub>
    <mrow>
      <mo fence="true" form="prefix">(</mo>
      <msub>
        <mi>I</mi>
        <mtext>enc</mtext>
      </msub>
      <mo>+</mo>
      <msub>
        <mi> ε </mi>
        <mn>0</mn>
      </msub>
      <mfrac>
        <mrow>
          <mi mathvariant="normal">d</mi>
        </mrow>
        <mrow>
          <mrow>
            <mi mathvariant="normal">d</mi>
          </mrow>
          <mi>t</mi>
        </mrow>
      </mfrac>
      <msub>
        <mo movablelimits="false">∫</mo>
        <mi>S</mi>
      </msub>
      <mover>
        <mi>E</mi>
        <mo stretchy="false" style="transform:scale(0.75) 
        translate(10%, 30%);">→</mo>
      </mover>
      <mo>∘</mo>
      <mover>
        <mi>n</mi>
        <mo stretchy="false" style="math-style:normal;
        math-depth:0;">^</mo>
      </mover>
      <mspace width="0.2778em"></mspace>
      <mrow>
        <mi mathvariant="normal">d</mi>
      </mrow>
      <mi>a</mi>
      <mo fence="true" form="postfix">)</mo>
    </mrow>
  </mrow>
</math>
\end{verbatim}